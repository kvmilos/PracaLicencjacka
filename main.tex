\documentclass[licencjacka]{pracamgr}
\usepackage{setspace}
\usepackage{hyperref}
\onehalfspacing
\autor{Kamil Tomaszek}{432044}
\title{Minimalizacja długości zależności w strukturach współrzędnie złożonych: badanie korpusowe na podstawie Polish Dependency Bank}
\kierunek{Kognitywistyka}
\opiekun{dr hab. Adama Przepiórkowskiego\\
Instytut Podstaw Informatyki PAN}
\date{Czerwiec 2023}
\keywords{koordynacja, minimalizacja długości zależności, Polish Dependency Bank, drzewo zależnościowe, korpus języka polskiego}

\begin{document}

\maketitle

\tytulang{Dependency Length Minimisation in coordinate structures: A corpus study based on \\ Polish Dependency Bank}

\begin{abstract}
Tekst streszczenia
\end{abstract}

\thispagestyle{empty}
\setcounter{page}{3}
\tableofcontents 

\chapter{Wstęp}
\section{Motywacja i cel pracy}
Tekst sekcji
\section{Zakres i struktura pracy}
Tekst sekcji

\chapter{Teoretyczne podstawy minimalizacji długości zależności w strukturach współrzędnie złożonych}
Tekst rozdziału
\section{Koordynacja w języku polskim}
Tekst sekcji
\section{Zarys teorii zależności składniowej}
Tekst sekcji
\section{Minimalizacja długości zależności w koordynacji}
Tekst sekcji
\section{Co poszczególne reprezentacje przewidują}
Tekst sekcji

\chapter{Polish Dependency Bank -- źródło danych i narzędzi do analizy}
Tekst rozdziału
\section{Krótki opis Polish Dependency Bank}
Tekst sekcji
\section{Preprocessing danych}
Tekst sekcji
\section{Dane po preprocessingu}
Tekst sekcji

\chapter{Analiza statystyczna}
Tekst rozdziału
\section{Hipoteza, metody}
Tekst sekcji
\section{Przedstawienie wyników analizy statystycznej w R}
Tekst sekcji
\section{Testowanie hipotez}
Tekst sekcji

\chapter{Dyskusja wyników}
Tekst rozdziału
\section{Podsumowanie wyników badań}
Tekst sekcji
\section{Interpretacja wyników}
Tekst sekcji
\section{Przegląd literatury}
Tekst sekcji

\chapter{Zakończenie}
Tekst rozdziału
\section{Podsumowanie pracy i wnioski}
Tekst sekcji
\section{Perspektywy dalszych badań}
Tekst sekcji

\begin{thebibliography}{99}
\addcontentsline{toc}{chapter}{Bibliografia}

% \bibitem[Bea65]{beaman} Juliusz Beaman, \textit{Morbidity of the Jolly
%     function}, Mathematica Absurdica, 117 (1965) 338--9.

% \bibitem[Blar16]{eb1} Elizjusz Blarbarucki, \textit{O pewnych
%     aspektach pewnych aspektów}, Astrolog Polski, Zeszyt 16, Warszawa
%   1916.

% \bibitem[Fif00]{ffgg} Filigran Fifak, Gizbert Gryzogrzechotalski,
%   \textit{O blabalii fetorycznej}, Materiały Konferencji Euroblabal
%   2000.

% \bibitem[Fif01]{ff-sr} Filigran Fifak, \textit{O fetorach
%     $\sigma$-$\rho$}, Acta Fetorica, 2001.

% \bibitem[Głomb04]{grglo} Gryzybór Głombaski, \textit{Parazytonikacja
%     blabiczna fetorów --- nowa teoria wszystkiego}, Warszawa 1904.

% \bibitem[Hopp96]{hopp} Claude Hopper, \textit{On some $\Pi$-hedral
%     surfaces in quasi-quasi space}, Omnius University Press, 1996.

% \bibitem[Leuk00]{leuk} Lechoslav Leukocyt, \textit{Oval mappings ab ovo},
%   Materiały Białostockiej Konferencji Hodowców Drobiu, 2000.

% \bibitem[Rozk93]{JR} Josip A.~Rozkosza, \textit{O pewnych własnościach
%     pewnych funkcji}, Północnopomorski Dziennik Matematyczny 63491
%   (1993).

% \bibitem[Spy59]{spyrpt} Mrowclaw Spyrpt, \textit{A matrix is a matrix
%     is a matrix}, Mat. Zburp., 91 (1959) 28--35.

% \bibitem[Sri64]{srinis} Rajagopalachari Sriniswamiramanathan,
%   \textit{Some expansions on the Flausgloten Theorem on locally
%     congested lutches}, J. Math.  Soc., North Bombay, 13 (1964) 72--6.

% \bibitem[Whi25]{russell} Alfred N. Whitehead, Bertrand Russell,
%   \textit{Principia Mathematica}, Cambridge University Press, 1925.

% \bibitem[Zen69]{heu} Zenon Zenon, \textit{Użyteczne heurystyki
%     w~blabalizie}, Młody Technik, nr~11, 1969.

\end{thebibliography}

\chapter*{Załączniki}
\addcontentsline{toc}{chapter}{Załączniki}
\end{document}
