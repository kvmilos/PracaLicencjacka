\documentclass[licencjacka]{pracamgr_Kogni}
\usepackage{setspace}
\usepackage{natbib}
\usepackage{xcolor}
\usepackage{hyperref} 
 \hypersetup{ 
     colorlinks=true, 
     linkcolor=black, 
     citecolor=black,       
     urlcolor=darkgray, 
     }
\onehalfspacing
\autor{Kamil Tomaszek}{432044}
\title{Minimalizacja długości zależności w~strukturach współrzędnie złożonych: badanie korpusowe na~podstawie Polish Dependency Bank}
\kierunek{kognitywistyka}
\opiekun{\\ 
\bfseries prof. dr hab. Adama Przepiórkowskiego\\
Uniwersytet Warszawski}
\date{czerwiec 2023}
\keywords{koordynacja, minimalizacja długości zależności, Polish Dependency Bank, drzewo zależnościowe, korpus języka polskiego}

\begin{document}

\maketitle

\tytulang{Dependency Length Minimisation in coordinate structures: A corpus study based on \\ Polish Dependency Bank}

\begin{abstract}
Praca licencjacka na temat "Minimalizacja długości zależności w strukturach współrzędnie złożonych: badanie korpusowe na podstawie Polish Dependency Bank"  jest poświęcona zjawisku minimalizacji długości zależności (DLM) w koordynacji w języku polskim. Celem pracy jest sprawdzenie hipotez na ten temat oraz przedstawienie dodatkowych analiz. Ma ona charakter empiryczny i opiera się na danych pochodzących z Polish Dependency Bank (PDB). Praca składa się z sześciu rozdziałów. W pierwszym rozdziale przedstawiłem motywację, cel i zakres pracy oraz jej strukturę. W drugim rozdziale omówiłem teoretyczne podstawy pracy, tj. reprezentacje koordynacji w języku polskim, teorię zależności składniowej i DLM w koordynacji. W trzecim rozdziale opisałem źródło danych i narzędzia do analizy, tj. PDB i preprocessing
danych za pomocą algorytmu napisanego w Pythonie. W czwartym rozdziale zaprezentowałem wyniki analizy statystycznej wykonanej w R oraz testowanie hipotez za pomocą m. in. testu chi-kwadrat. W piątym rozdziale dokonałem dyskusji wyników, interpretacji ich znaczenia i porównania z literaturą naukową. W szóstym rozdziale podsumowałem pracę i wnioski oraz zaproponowałem perspektywy dalszych badań.
\end{abstract}

\thispagestyle{empty}
\setcounter{page}{3}
\tableofcontents 

\chapter{Wstęp}
W tym rozdziale przedstawiam motywację i cel pracy licencjackiej na temat "Minimalizacja długości zależności w strukturach współrzędnie złożonych: badanie korpusowe na podstawie Polish Dependency Bank", a także omawiam jej zakres oraz strukturę.

\section{Motywacja i cel pracy}
Praca ta ma na celu analizę zjawiska minimalizacji długości zależności – DLM (z ang. Dependency Length Minimisation), czyli tendencji do umieszczania elementów współrzędnych o różnych długościach w sposób, by zmniejszyć odległość zarówno między nimi samymi, jak i między nimi a innymi elementami zdania, w koordynacjach w języku polskim. Koordynacja to zjawisko, gdy wiele części zdania ma jeden nadrzędnik i każda z nich się z nim koordynuje. Zjawisko to jest istotne dla teorii składniowej i reprezentacji językowych, ponieważ dotyczy zarówno formy jak i znaczenia zdań. W pracy tej sprawdzono dwie hipotezy dotyczące długości członów w koordynacjach w języku polskim: 1. że dłuższy człon koordynacji jest częściej ze strony prawej i 2. że dłuższy człon koordynacji jest częściej dalej od jej nadrzędnika.
\\

(0) \textit{Widziałem} Asię \textbf{i} jej śmiesznego, młodszego brata. \\
\\
Długości członów mierzono na cztery różne sposoby, licząc znaki, sylaby, słowa oraz tokeny. W przykładzie (0) odpowiednie wartości wynosiłyby [4~vs~31, 2~vs~9, 1~vs~4, 1~vs~5]. Szybko pokazano, że jedna z hipotez zachodzi w większości przypadków, więc następnie omówiono wpływ obecności i pozycji nadrzędnika oraz długości różnicy między analizowanymi członami na proporcje danych, w których hipoteza ta jest prawdziwa. Praca ta ma charakter empiryczny, opiera się na danych pochodzących z Polish Dependency Bank (PDB), czyli korpusu języka polskiego zawierającego ponad 22 tysiące drzew zależnościowych oraz na podobnej pracy badającej te same zależności, ale dla języka angielskiego \citep{AnonimoweNieopublikowane}.

\section{Zakres i struktura pracy}
Praca składa się z sześciu rozdziałów. W rozdziale drugim omówiono teoretyczne podstawy pracy, tj. przedstawiono czym są koordynacje -- na przykładzie języka polskiego, opisano zarys teorii zależności składniowej, zaprezentowano . W rozdziale trzecim opisano źródło danych, czyli Polish Dependency Bank, jak i ich preprocessing -- działanie algorytmu, napisanego w języku Python, wybierającego koordynacje oraz informacje o nich z PDB, a także pokazano format danych po preprocessingu w pliku z rozszerzeniem ".csv". W rozdziale czwartym zaprezentowano hipotezy badawcze, ich testowanie oraz analizy statystyczne w R, między innymi test Wilcoxona, testy chi-kwadrat oraz ogólne modele liniowe (GLM – z ang. Generalised Linear Models). W rozdziale piątym omówiono wyniki badań i ich interpretację w kontekście istniejącej wcześniej literatury naukowej. W rozdziale szóstym podsumowano pracę, wyciągnięto z niej wnioski oraz zaproponowano perspektywy dalszych badań. 

\chapter{Teoretyczne podstawy minimalizacji długości zależności w strukturach współrzędnie złożonych}
W tym rozdziale omawiam teoretyczne podstawy pracy, tj. reprezentacje koordynacji w języku polskim, teorię zależności składniowej i DLM w koordynacji.

\section{Koordynacja w języku polskim}
Zacznę od przedstawienia pojęcia koordynacji. Koordynacja to zjawisko w językach naturalnych, które zachodzi w struktruach złożonych -- zarówno współrzędnie, jak i podrzędnie. Polega na zestawieniu dwóch lub więcej elementów o tej samej funkcji składniowej za pomocą spójników lub interpunkcji i tym samym złączenie ich w jeden, większy element, zachowujący te same funkcje składniowe. Jest ono jednym z podstawowych sposobów łączenia słów, czy zdań. Elementami koordynacji mogą być zarówno pojedyncze słowa (1a, 1b),  wyrażenia (1c), jak i całe zdania (1d):
\\

(1)

a. Ania \textbf{i} Julia \textit{idą} na spacer.

b. Ania \textbf{i} Julia.

c. Wesoła Marysia \textbf{oraz} smutny Janek \textit{wybrali się} do parku.

d. \textit{Kuba} zjadł obiad \textbf{a} następnie poszedł spać.
\\
\\
Człony koordynacji nazywamy koordynantami, to co je łączy -- spójnikiem koordynacji (w przykładach w tej pracy jest on ilustrowany pogrubionym tekstem), a wyraz nadrzędny względem obu członów -- głową koordynacji (w przykładach ilustrowany kursywą). Jak widać w (1b) nie zawsze istnie jengłowa koordynacji.
W podanych wyżej przykładach koordynantami są: (1a, 1b) Ania, Julia; (1c) Wesoła Marysia, smutny Janek; (1d) zjadł obiad, poszedł spać.

Ze względów semantycznych zwykle wyróżnia się cztery rodzaje koordynacji: koordynacje koniunkcyjne (2a), koordynacje dysjunkcyjne (2b), koordynacje adwersatywne (2c) oraz koordynacje kauzalne (2d) \citep{Haspelmath2007}. Każde z nich używają różnych zestawów spójników, które łączą koordynanty. W koordynacjach koniunkcyjnych koordynanty łączą m. in. spójniki [i, oraz, ani, tudzież, również], a w koordynacjach dysjunkcyjnych -- [albo, bądź, lub, czy], lecz w obu tych kategoriach wykorzystywana jest także interpunkcja. W koordynacjach adwersatywnych używamy m. in. spójników [ale, lecz, zaś, natomiast, jednak], a w koordynacjach kauzalnych -- [bo, ponieważ, dlatego że]. W tej pracy skupię się na pierwszych trzech rodzajach koordynacji, jako że to one występują w strukturach współrzędnie złożonych.
\\

(2)

a. Marta \textit{zjadła} jabłko \textbf{i} gruszkę.

b. Ona miała szesnaście \textbf{lub} siedemnaście \textit{lat}.

c. \textit{Byli} ładni, \textbf{ale} głupi.

d. Nie zrobiłem pracy domowej, \textbf{bo} nie chciałem.

\section{Zarys teorii zależności składniowej}


\section{Minimalizacja długości zależności w koordynacji}
Tekst sekcji
\section{Co poszczególne reprezentacje przewidują}
Tekst sekcji

\chapter{Polish Dependency Bank -- źródło danych i narzędzia do analizy}
Tekst rozdziału
\section{Krótki opis Polish Dependency Bank}
Tekst sekcji
\section{Preprocessing danych}
Tekst sekcji
\section{Dane po preprocessingu}
Tekst sekcji

\chapter{Analiza statystyczna}
Tekst rozdziału
\section{Hipoteza, metody}
Tekst sekcji
\section{Przedstawienie wyników analizy statystycznej w R}
Tekst sekcji
\section{Testowanie hipotez}
Tekst sekcji

\chapter{Dyskusja wyników}
Tekst rozdziału
\section{Podsumowanie wyników badań}
Tekst sekcji
\section{Interpretacja wyników}
Tekst sekcji
\section{Przegląd literatury}
Tekst sekcji

\chapter{Zakończenie}
Tekst rozdziału
\section{Podsumowanie pracy i wnioski}
Tekst sekcji
\section{Perspektywy dalszych badań}
Tekst sekcji

\begin{thebibliography}{99}
\addcontentsline{toc}{chapter}{Bibliografia}

\bibitem [Anonimowe Zgłoszenie na ACL(nieopublikowane)]{AnonimoweNieopublikowane}
Anonimowe Zgłoszenie na ACL 2023 (nieopublikowane) Conjunct Lengths in English, Dependency Length Minimization, and Dependency Structure of Coordination

\bibitem [Haspelmath(2007)]{Haspelmath2007}
Haspelmath, M. (2007) Coordination, \textit{Language Typology and Syntactic Description, Volume II: Complex constructions}, 1-51

\end{thebibliography}


\chapter*{Załączniki}
\addcontentsline{toc}{chapter}{Załączniki}
\end{document}
